\documentclass[11pt]{article}
\usepackage[margin=1in]{geometry}
\usepackage{graphicx}
\usepackage{booktabs}
\usepackage{multirow}
\usepackage{amsmath}
\usepackage{tikz}
\usetikzlibrary{arrows.meta, positioning}
\usepackage{float}
\usepackage{hyperref}
\usepackage{enumitem}
\usepackage{fontspec}
\setmainfont{Noto Sans CJK SC}

\title{AgriFrost-AI 霜冻风险预测进展报告(中文版)}
\author{AgriFrost-AI 团队}
\date{\today}

\begin{document}
\sloppy
\XeTeXlinebreaklocale "zh"
\XeTeXlinebreakskip = 0pt plus 0.1pt
\setlength{\emergencystretch}{3em}
\setlength{\parindent}{2em}
\maketitle

\begin{abstract}
本文档汇总 AgriFrost-AI 在 F3 Innovate 霜冻风险预测挑战中的阶段性成果。我们基于 2010--2025 年 18 个 CIMIS 站点的逐小时观测,构建了端到端的数据清洗、特征构建、矩阵化建模与跨站验证流程,覆盖 3、6、12、24 小时四个预报窗口。报告重点阐述数据处理策略、ABCD 特征矩阵、模型组合、实验结果、校准表现、LOSO 泛化能力、关键近地特征以及与种植者决策的衔接,并展望 ERA5/HRRR 同步气象信息的融入方向。
\end{abstract}

\tableofcontents
\clearpage
\section{引言}
霜冻依旧是加州高价值作物面临的主要气象风险,尤其在花期阶段,单次强辐射霜冻即可造成巨大经济损失。F3 Innovate 挑战旨在评估数据驱动方法能否在多地形、多气候条件下提供可靠的短期霜冻预测,并要求参赛队伍完成以下任务:
\begin{itemize}[leftmargin=*]
    \item 同时输出 3、6、12、24 小时的霜冻概率与温度预估;
    \item 量化概率校准质量,包括 Brier、ECE、可靠性图;
    \item 通过留一站(LOSO)检验空间泛化,避免站点泄漏;
    \item 解释概率在实际防护决策中的使用方式。
\end{itemize}
AgriFrost-AI 项目遵循上述要求构建了可复现的端到端流程,并在本报告中详细介绍方法与实验结果。

\section{数据描述}
\subsection{观测来源与覆盖范围}
\begin{itemize}[leftmargin=*]
    \item \textbf{主数据集}:本研究使用的逐小时气象观测来自加州灌溉管理信息系统(California Irrigation Management Information System, CIMIS),覆盖加利福尼亚中央谷地及周边山麓地区的 18 个站点。站点沿中央谷地呈南北向分布(图 \ref{fig:stationmap_cn}),从 Sacramento 平原延伸至 Bakersfield 区域,跨越冷空气堆积易发区、地势抬升带和高蒸散农田带等多样微气候环境。数据时间跨度为 2010--2025 年,共计约 236 万条逐小时记录,其中霜冻事件(温度低于 \(0~^\circ\mathrm{C}\))占比接近 0.9\%,为构建高分辨率霜冻预测模型提供了连续而丰富的长期序列。原始数据与配套脚本统一托管在 \href{https://github.com/CarlSaganPhD/frost-risk-forecast-challenge}{GitHub 仓库},便于版本追踪与复现。
    \item \textbf{站点元数据}:包含站号、名称、CIMIS 区域、县市、经纬度、海拔、GroundCover、启停日期及是否为 ETo 站等信息,用于空间聚合与 LOSO 分组;完整列表见 Supplementary Table~1。
    \item \textbf{外部数据}:可选地接入 ERA5 或 HRRR 的冷空气平流、云量与辐射变量,以捕捉大尺度驱动因子,见第 \ref{sec:synoptic_cn} 节。
\end{itemize}

\begin{figure}[H]
    \centering
    \includegraphics[width=0.72\textwidth]{figures/station_distribution_map.png}
    \caption{18 个 CIMIS 站点空间分布(静态图);交互式版本见 Supplementary Materials}
    \label{fig:stationmap_cn}
\end{figure}

\subsection{处理流程}
\begin{enumerate}[leftmargin=*]
    \item \textbf{数据汇聚}:合并各站点 CSV/Parquet 文件,统一转为本地太阳时,并附加站点元信息。
    \item \textbf{质量控制}:按照变量阈值剔除异常值、使用插值填补短缺测,并保留缺失掩码。
    \item \textbf{时间上下文}:构建 1/3/6/12/24 小时的滞后和滑动统计特征,同时加入日内与年周期谐波。
    \item \textbf{空间上下文}:依据不同半径(20--200 km)的邻站集合,计算均值、极值、方差、梯度及距离加权统计,共生成 288 个邻域特征。
    \item \textbf{标签生成}:一次性生成 4 个时间跨度的霜冻二分类标签及温度回归目标,确保所有训练任务使用同一套标签。
\end{enumerate}

\subsection{数据质量与 QC 概览}
统一的 `DataCleaner` 流程首先解析所有以 \texttt{qc} 开头的质量字段,并依据 CIMIS 标准仅保留 “空白/通过” 与 “Y” 两类标记;其余(M、R、S、Q、P 及其他维护标志)全部转为缺失,再结合哨兵值剔除与按站点前向填补,保证训练集无伪信号泄漏。2010--2025 年的 2,367 万条 QC 标记中,98.29\% 属于 “可用” 档,0.62\% 为缺测(M),0.45\% 被拒绝(R),0.20\% 属于严重离群(S),其余仅 1\% 内;图 \ref{fig:qc_pie_cn} 展示了整体分布,而图 \ref{fig:qc_station_cn} 以堆叠柱状将各站点的全部 QC 结构展开,便于识别是否存在持续异常的设备或环境。

\begin{figure}[H]
    \centering
    \includegraphics[width=0.62\textwidth]{figures/qc_flag_distribution.png}
    \caption{CIMIS QC 标记分布(2010--2025),基于 18 个站点全部逐小时观测}
    \label{fig:qc_pie_cn}
\end{figure}

\begin{figure}[H]
    \centering
    \includegraphics[width=0.75\textwidth]{figures/qc_flag_distribution_per_station.png}
    \caption{各站点 QC 构成,使用堆叠柱状展示 “可用/缺测/拒绝/严重”等占比}
    \label{fig:qc_station_cn}
\end{figure}

\subsection{ABCD 特征矩阵}
特征设计遵循 “单站/多站 × 原始/工程” 的 2×2+1 框架,表 \ref{tab:matrix_cn} 总结了四个主矩阵:

\begin{table}[H]
    \centering
    \caption{ABCD 特征矩阵概览}
    \label{tab:matrix_cn}
    \begin{tabular}{p{1.2cm}p{1.8cm}p{7.8cm}p{2.8cm}}
        \toprule
        矩阵 & 空间范围 & 特征构成 & 典型模型 \\
        \midrule
        A & 单站 & 12 个原始 CIMIS 变量 + 周期编码 & LightGBM, CatBoost \\
        B & 单站 & 175 个精选工程特征(滞后、滑动、异常度) & LightGBM, XGBoost, CatBoost \\
        C & 半径聚合 & 原始 + 288 个邻站统计 + 缺失掩码 & LightGBM, ST-GCN, DCRNN \\
        D & 半径聚合 + 工程 & B 矩阵特征 + 邻站特征(534+ 维) & CatBoost, XGBoost, 集成 \\
        \bottomrule
    \end{tabular}
\end{table}

其中矩阵 C 表现最优,保留 12 个本地观测与 288 个邻域统计,并通过特征重要性分析确认:邻域土壤温度梯度、露点差、蒸汽压亏缺是最具早期预警价值的组合;矩阵 B 受滞后统计主导,而矩阵 A 仅依赖原始变量,在夜间逆温条件下容易失效;矩阵 D 虽然叠加了工程特征,但校准收益有限,提示仍需引入更高质量的同步气象指标。

\subsection{划分与防泄漏}
所有模型采用 70\% 训练、15\% 验证、15\% 测试的时间顺序划分,并保持站内时间单调。LOSO 评估在每次迭代中剔除一个站点、使用剩余站点拟合全部预处理和模型,再在被剔除站点上测试,以检验空间泛化。任何缩放、PCA、邻域构建等步骤均仅基于训练子集,保证无未来信息泄漏。

\section{方法}
\subsection{流程概览}
\begin{figure}[H]
    \centering
    \begin{tikzpicture}[
        process/.style={rectangle, rounded corners, draw=black, fill=gray!10, minimum width=3.4cm, minimum height=1.1cm},
        datastore/.style={rectangle, rounded corners, draw=black, fill=gray!15, minimum width=3.4cm, minimum height=1.1cm},
        line/.style={-Latex, thick}
    ]
    \node[datastore] (ndp) {数据与元数据};
    \node[process, right=1.6cm of ndp] (prep) {清洗与质控};
    \node[process, right=1.8cm of prep] (features) {时空特征构建};
    \node[process, below=1.6cm of features] (models) {模型族训练};
    \node[process, left=1.8cm of models] (calib) {校准与可靠性};
    \node[process, left=1.8cm of calib] (eval) {LOSO 与多视角评估};
    \node[process, below=1.6cm of calib] (decision) {决策支持输出};
    \draw[line] (ndp) -- (prep);
    \draw[line] (prep) -- (features);
    \draw[line] (features) -- (models);
    \draw[line] (models) -- (calib);
    \draw[line] (calib) -- (eval);
    \draw[line] (models) -- (decision);
    \draw[line] (eval) |- (decision);
    \end{tikzpicture}
    \caption{AgriFrost-AI 方法流程}
\end{figure}

\subsection{模型组合与实验设置}
\begin{itemize}[leftmargin=*]
    \item \textbf{树模型}:LightGBM(主力)、XGBoost、CatBoost,学习率 0.05,树数 200--1000,深度 6--8,并依据霜冻稀缺性调整正负样本权重。
    \item \textbf{图/序列模型}:ST-GCN、DCRNN、GRU/LSTM,用于检验显式时空建模对比工程特征的增益。
    \item \textbf{实验调度}:统一 CLI 负责数据处理、训练、评估与产出归档。半径与时间窗口扫描采用固定随机种子,便于横向对比。
    \item \textbf{度量记录}:每次运行都会输出标准化指标、校准图、预测分布与配置快照,随后汇总至全局结果表进行分析。
\end{itemize}

\subsection{评估指标}
分类和回归双任务的评估指标包括 ROC-AUC、PR-AUC、Brier Score、期望校准误差(ECE)、温度 MAE/RMSE 与 \(R^2\)。其中 PR-AUC 用于衡量极度不平衡的霜冻事件判别力,Brier/ECE 评估概率是否可直接用于运营决策。

\section{实验结果}\label{sec:experiments_cn}
\subsection{整体概率性能}
矩阵 C + LightGBM 的最佳配置在各个预测窗口的表现见表 \ref{tab:prob_cn}。

\begin{table}[H]
    \centering
    \caption{霜冻概率与温度预测表现(时间留后 15\% 测试集)}
    \label{tab:prob_cn}
    \begin{tabular}{lcccccc}
        \toprule
        预测窗口 & ROC-AUC & PR-AUC & Brier & ECE & RMSE (\(^\circ\mathrm{C}\)) & MAE (\(^\circ\mathrm{C}\)) \\
        \midrule
        3 小时(100 km) & 0.9972 & 0.7282 & 0.0026 & 0.0012 & 1.54 & 1.16 \\
        6 小时(100 km) & 0.9936 & 0.5838 & 0.0036 & 0.0021 & 2.10 & 1.60 \\
        12 小时(200 km) & 0.9901 & 0.4914 & 0.0043 & 0.0032 & 2.42 & 1.85 \\
        24 小时(160 km) & 0.9877 & 0.4596 & 0.0044 & 0.0034 & 2.41 & 1.85 \\
        \bottomrule
    \end{tabular}
\end{table}

短期模型的判别力接近完美(ROC-AUC \(>0.99\)),PR-AUC 在 24 小时仍保持 0.46,说明即便霜冻极少也能有效排序。半径扫描显示:短期更倾向 60--120 km 的紧邻区域,而 12--24 小时则需要更大的邻域获取大尺度信号。

\subsection{校准与可靠性}
所有预测窗口的 Brier Score 均低于 0.005,ECE 低于 0.004。图 \ref{fig:reliability_cn} 展示了 3 小时模型的可靠性图,概率刻度整体贴近对角线,仅在高概率区略显保守。

\begin{figure}[H]
    \centering
    \includegraphics[width=0.7\textwidth]{figures/reliability_diagram_3h.png}
    \caption{3 小时模型的可靠性图(矩阵 C,半径 100 km)}
    \label{fig:reliability_cn}
\end{figure}

\subsection{LOSO 空间泛化}
LOSO 结果汇总于表 \ref{tab:loso_cn}。可见 ROC-AUC 在所有时间窗口均超过 0.987,且与常规测试集相比没有明显下降,说明邻域特征有效捕捉了跨站气候信息。

\begin{table}[H]
    \centering
    \caption{LOSO 与常规评估对比(18 个站点平均)}
    \label{tab:loso_cn}
    \begin{tabular}{lcccc}
        \toprule
        预测窗口 & ROC-AUC (标准) & ROC-AUC (LOSO) & 差值 (百分点) & MAE\(_{\text{LOSO}}\) (\(^\circ\mathrm{C}\)) \\
        \midrule
        3 小时 & 0.9965 & 0.9974 & +0.09 & 1.14 \\
        6 小时 & 0.9926 & 0.9938 & +0.12 & 1.55 \\
        12 小时 & 0.9892 & 0.9905 & +0.13 & 1.79 \\
        24 小时 & 0.9843 & 0.9878 & +0.35 & 1.93 \\
        \bottomrule
    \end{tabular}
\end{table}

\subsection{矩阵与模型对比}\label{sec:matrixcomparison_cn}
我们将所有实验按矩阵和模型族汇总,如表 \ref{tab:matrixbest_cn} 与图 \ref{fig:matrixcomparison_cn} 所示。Matrix C 在所有时间窗口均取得最优 Brier 与 PR-AUC;对比同一模型在不同矩阵时,可看到 LightGBM 从 A 到 C 的 ROC-AUC 提升约 0.010,突显空间特征的重要性。

\begin{table}[H]
    \centering
    \caption{每个矩阵/预测窗口的最佳配置}
    \label{tab:matrixbest_cn}
    \begin{tabular}{lcccccc}
        \toprule
        矩阵 & 预测窗口 & 模型 & 半径 (km) & ROC-AUC & PR-AUC & Brier \\
        \midrule
        A & 3h & LightGBM & 0 & 0.9967 & 0.7148 & 0.0027 \\
        A & 6h & LightGBM & 0 & 0.9923 & 0.5397 & 0.0041 \\
        A & 12h & LightGBM & 0 & 0.9856 & 0.3884 & 0.0049 \\
        A & 24h & CatBoost & 0 & 0.9284 & 0.0900 & 0.0050 \\
        B & 3h & LightGBM & 0 & 0.9969 & 0.7042 & 0.0029 \\
        B & 6h & LightGBM & 0 & 0.9937 & 0.5531 & 0.0038 \\
        B & 12h & LightGBM & 0 & 0.9896 & 0.4337 & 0.0044 \\
        B & 24h & CatBoost & 0 & 0.9392 & 0.1072 & 0.0049 \\
        C & 3h & LightGBM & 100 & 0.9972 & 0.7282 & 0.0026 \\
        C & 6h & LightGBM & 100 & 0.9936 & 0.5838 & 0.0036 \\
        C & 12h & LightGBM & 200 & 0.9901 & 0.4914 & 0.0043 \\
        C & 24h & LightGBM & 160 & 0.9877 & 0.4596 & 0.0044 \\
        D & 3h & CatBoost & 200 & 0.9874 & 0.3931 & 0.0038 \\
        D & 6h & CatBoost & 160 & 0.9718 & 0.2676 & 0.0043 \\
        D & 12h & CatBoost & 180 & 0.9604 & 0.1891 & 0.0046 \\
        D & 24h & CatBoost & 180 & 0.9467 & 0.1503 & 0.0047 \\
        \bottomrule
    \end{tabular}
\end{table}

\begin{figure}[H]
    \centering
    \includegraphics[width=0.8\textwidth]{figures/matrix_model_comparison.png}
    \caption{不同模型族在四个特征矩阵中的最佳 ROC-AUC 对比}
    \label{fig:matrixcomparison_cn}
\end{figure}

\section{特征洞察}
从矩阵 C 的特征重要性与半径消融实验可以提炼出以下结论:
\begin{itemize}[leftmargin=*]
    \item \textbf{土壤温度梯度}:邻域土温的最小值与梯度能反映冷空气积聚,是 3/6 小时模型的核心信号。
    \item \textbf{湿度与露点差}:露点/空气温差、蒸汽压亏缺在辐射霜冻形成前数小时会快速扩大,是 12/24 小时预警的重要指标。
    \item \textbf{谐波特征}:日内与年周期对上面信号起到 gating 作用,可显著降低假警报。
    \item \textbf{多变量组合}:同时监测土壤温度梯度、露点差、蒸汽压亏缺与相对湿度梯度,可使 PR-AUC 相比单站基线提升 36.7\%。
\end{itemize}

\section{决策支持}
\label{sec:decision_cn}
Calibrated 概率直接对应实际操作阈值:
\begin{itemize}[leftmargin=*]
    \item \textbf{20\% 概率}:加强传感器监控,提前检查风机/喷灌设备但尚无须启动。
    \item \textbf{50\% 概率}:启动灌溉系统预热、调配移动热源,并通知班组准备夜间值守。
    \item \textbf{80\% 概率}:在预报最低温前 1--2 小时启动风机或微喷,重点保护预测温度将低于 \(0~^\circ\mathrm{C}\) 的地块。
\end{itemize}
鉴于 Brier Score 均低于 0.005,这些概率阈值可以直接写入农场 SOP。

\section{同步气象拓展}\label{sec:synoptic_cn}
为进一步提升长时段性能,计划引入以下 ERA5/HRRR 特征:
\begin{enumerate}[leftmargin=*]
    \item 925--850 hPa 温度平流,用于衡量冷空气输送。
    \item 云量与长波下行辐射,刻画辐射冷却条件。
    \item 地表净辐射与土壤湿度,补充能量收支信息。
\end{enumerate}
改造方案为:在同一配置下分别训练 “仅地面” 与 “地面+同化” 模型,通过 LOSO 成对 t 检验验证增益。

\section{可复现性}
项目使用声明式配置与固定随机种子,所有实验目录均包含:原始参数、数据切分信息、训练日志、指标文件、可靠性图与模型权重。手稿由同一仓库直接编译而得,避免了报告与代码脱节。

\section{结论与展望}
\begin{itemize}[leftmargin=*]
    \item 短期模型(3/6 小时)在判别力与校准方面已满足挑战要求,PR-AUC 与 Brier 均处于行业领先水平。
    \item Matrix C 的邻域特征显著提高了空间泛化能力,LOSO 几乎无性能折损。
    \item 下一阶段工作包括:同化 ERA5/HRRR 特征、完成所有矩阵的 LOSO 细粒度报告,以及对特征重要性进行站点级解释,帮助种植者理解不同区域的主导因子。
\end{itemize}

\end{document}

